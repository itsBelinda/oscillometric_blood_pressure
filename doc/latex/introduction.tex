\chapter{Introduction}\label{cp:Intro}
%Explain the background to the project and the reasons why your particular piece of work was considered worthwhile. This leads to the aims of the project: what you are trying to achieve. Be specific.
\section{Problem Description}
Traditional blood pressure measurements require an inflatable cuff that is fixed tightly around the upper arm of a patient. Pressure in the cuff is increased above the value of the expected blood pressure and slowly released while listening to the heart sounds on a stethoscope to determine blood pressure. This process is performed by a medical professional. In recent years, automatic tools have eliminated the dependency on professionals and made taking one's blood pressure at home possible and convenient. 

Most automatic devices use small changes in pressure during deflation to detect blood pressure. This is the oscillometric method. It has been around for more than 40 years but has not improved much since its beginnings. Its limitations are known, and many proposed algorithms have tried to address them since. Unfortunately, without notable success. Besides, commercially available devices use proprietary, closed-source algorithms that can not be scientifically challenged.


\section{Objectives}

This project aims to examine existing and proposed algorithms for oscillometric blood pressure measurement. They will be judged based on robustness, underlying theory and applicability in an automatic real-time system.

The most promising identified algorithm is implemented using the provided hardware on a Linux system. The hardware consists of a simple, manual blood pressure cuff equipped with a pressure sensor that is connected to an analogue to digital converter (ADC). This is connected to a Linux computer. The implementation is done in C++ as an open-source project that is available on GitHub. The software is designed so the algorithm implementation can easily be adopted in other projects.

The developed application guides the user through taking a measurement and reports the results on the screen. Meanwhile, raw pressure data is recorded and stored in a file. Ideally, the project would have been concluded by performing a small set of experiments and building a database of measurements. Unfortunately, this was not possible in the current environment and the short time frame. 

\section{Outline}
This report is split up into three main parts, the theory, discussed in chapter \myref{cp:theory}, the implementation of the developed application, discussed in chapter \myref{cp:algo}, \myref{cp:hw} and \myref{cp:sw} and the discussion of the results in chapter \ref{cp:res}.

The first part, chapter \myref{cp:theory}, discusses the theory of oscillometric blood pressure measurements. It focusses on research material and algorithms suitable for implementation in real-time processing. It starts by explaining what blood pressure is and how it is can be accurately measured. Next, different algorithms for automatic oscillometric measurements are explained starting with the most popular ones and ending with proposed alternative methods. The chapter is concluded with a summary and the identified algorithm to implement in this project.

The second part, chapter \myref{cp:algo} discusses the implemented algorithm in detail. It is a description of the logic and principles upon which the software is built on.

Chapter \myref{cp:hw} defines briefly the used hardware, followed by chapter, \myref{cp:sw}. It discusses the implementation of the application that was developed. It focusses on the implemented architecture and structure of the application. The complete source code of the software is available on GitHub and documented using Doxygen.

The next part, chapter \myref{cp:res}, highlights what approaches worked well to determine blood pressure from oscillometric data and what caused difficulties. 

Finally, the last chapter, \myref{cp:concl} looks back at what was achieved in the project. Recommendations are given on further steps, how to develop the software and improve the algorithm. 

