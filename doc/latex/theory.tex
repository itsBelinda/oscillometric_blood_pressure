\chapter{Background}\label{cp:theory}
This chapter discusses the findings of researching existing and proposed algorithms for determining systolic and diastolic blood pressure through the oscillometric method. This report focusses on relatively simple algorithms that can be implemented in a real-time application, which is the goal of this project.

The first part of this chapter defines what blood pressure is and how it is measured manually. Afterwards
%TODO do I include python code showing why certain algorithms do not work or do I leave that be? --> discussion

A brief section at the end of this chapter explores algorithms beyond the scope of this project that have been proposed in recent literature.

\section{What is Blood Pressure?}
Blood pressure is an important bio-medical measurement and often used for diagnostics in cardiovascular diseases. Most commonly, systolic and diastolic blood pressure are mentioned in pairs, but what do those numbers mean?

The heart is a pump that has two phases. When the ventricle is relaxed, the heart fills up and the diastolic blood pressure (DBP) is observed. When the heart contracts, it pushes the blood through the arterial system and the systolic blood pressure (SBP) is observed. A third characteristic is the difference between SBP and DBP, the pulse pressure (PP). Finally, the mean arterial pressure (MAP) is defined as the average pressure in the artery. Figure \ref{fig:BP} shows how these pressure values are connected in one heartbeat. MAP is the area underneath the blue curve divided by the time of one pulse, indicated by the orange area. \cite{Boron2012}\ Because the blood pressure is simplified as a sine wave the MAP is in the middle between SDP and DBP, usually the MAP is closer to DBP. \paragraph{}

\begin{figure}[ht]
\centering
%\includegraphics[width=0.55\textwidth]{figures/MAP.pdf}
\caption{Blood pressure characteristic values. Blood pressure is simplified as a sine wave.} %TODO update picture: currently heart rate of 120 
\label{fig:BP}
\end{figure}


Blood pressure can be measured with invasive and non-invasive methods. The most accurate methods are invasive, but require professional expertise because a catheter has to be injected into the blood vessel. Subsequently, non-invasive methods are more common because they are more convenient to use. With automated methods, taking blood pressure at home has become convenient and easy. However, the inaccuracies of these measurements are often ignored or even unknown.

\section{Manual Blood Pressure Measurement}
The manual method of measuring blood pressure relies on a sphygmomanometer and listening to the heart or Korotkoff sounds (named after their discoverer) with a stethoscope. This is called the \emph{auscultatory method}. \emph{Palpation} is another method where the pulse is felt at the wrist. It only allows detection of the SBP. The auscultatory method is described below because it provides recommendations that are true for all blood pressure measurements using a cuff.

The sphygmomanometer is shown in figure \ref{fig:sphy}. It consists of an inflatable cuff that is connected to a rubber pump and a scale that indicates the pressure in the cuff in \SI{}{\mmHg}. The pump is equipped with a valve that can be opened to release air.


Millimetres of mercury (\SI{}{\mmHg}) is a unit to describe pressure. \SI{1}{\mmHg} is defined as the pressure generated by a column of mercury of \SI{1}{\mm} height. Millimetres of Mercury is not part of the International System of Units (SI) but was defined by it before 2019 as \SI{1}{\mmHg} = \SI{133.322}{\Pa} using standard gravity.\cite{SI2006} Since 2019, \SI{}{\mmHg} is no longer included in the SI brochure. However, it is still widely used in the medical field. \cite{Prazak2020}

\begin{figure}[h]
\centering
%\includegraphics[scale=0.125]{figures/spy.pdf}
\caption{Picture of a sphygmomanometer showing the cuff, inflation pump and mercury meter.}
\label{fig:sphy}
\end{figure}

Intra-arteria or direct BP measurement is considered the 'gold standard', but most studies compare the developed algorithm with the auscultatory method, which underestimated SBP and overestimates DBP. \cite{Sapinski1996} Nonetheless, the protocols to test automatic blood pressure monitors given by the British Hypertension Society (BHS) and their American counterpart, the Association for the Advancement of Medical Instrumentation (AAMI), both use the auscultatory method as a reference. \cite{Jazbinsek2010,OBrien1993,AAMI2013}

\subsection{Auscultatory Method}
The cuff size must be appropriate for the patient's arm. The rubber bladder inside the cuff should cover more than \SI{80}{\%} but less than \SI{100}{\%} of the arm's circumference.

The cuff is completely deflated and tightly fit around the upper arm of the patient. The centre of the bladder should be over the brachial artery and the arm supported at heart level. The stethoscope is to be placed over the artery between the cuff and the patient's elbow and should not touch the cuff. This could influence the reading because it puts additional pressure on the artery and can interfere with hearing the sounds.\cite{Lloyd2018,Reeves1995}

The valve at the pump is completely closed and the cuff is inflated to a pressure of about \SI{30}{\mmHg} above the expected systolic pressure. This causes the artery to flatten and not let any blood through. Subsequently, the valve is slightly opened to slowly deflate the cuff at a rate of approximately \SI{3}{\mmHg/\second}. The artery will open and let small amounts of blood through. Systolic pressure is read when a pulse is first heard. Meanwhile, deflation continues. Diastolic pressure is read when the sounds disappear completely. Ultimately, blood can flow freely through the artery. After making sure that no further sounds can be heard by deflating for at least another \SI{10}{\mmHg}, the valve is opened completely to rapidly empty the cuff.\cite{Lloyd2018,Reeves1995}

There are some discrepancies in literature on how to define the point of the diastolic blood pressure. Some references define it as when the Korotkoff sounds disappear completely. \cite{Lloyd2018,Reeves1995} Others, predominantly older resources, define it as the point where the sounds are muffled.\cite{Boron2012} Here, the former is assumed, because most current literature does and some literature suggests that the auscultatory method overestimates diastolic blood pressure. \cite{Chandrasekhar2019}

\subsection{Calculation of MAP}
Traditionally, MAP is calculated as an estimate by adding a third of the PP to the DBP. Or, how it is more commonly referred to, adding SBP and twice the DBP and dividing the result by 3 (equation \ref{eq:MAP}). According to calculations by Bos \textit{et al.}, this generally underestimates the MAP and they suggest to add $40\%$ rather than a third of DBP. Furthermore, they suggest to using MAP measured by oscillometric devices rather than calculating it from values obtained through the auscultatory method. \cite{Bos2007}

\begin{equation}
\label{eq:MAP}
MAP = \frac{SBP-DBP}{3}+DBP = \frac{SBP+2\times DBP}{3}
\end{equation}

According to Joe \cite{Joe2019}, nurses in intensive care units (ICU) today still use the manual calculation method on the automatically obtained SBP and DBP rather than rusting the measured MAP, which introduces large errors.

\section{Automatic Blood Pressure Measurement}

Most modern automatic devices use the oscillometric method to determine blood pressure. They are used both at home and in professional environments.

Automatic blood pressure monitors measure pressure in a cuff similar to the one used in the manual method and uses the small pressure changes (oscillations) extracted from the deflation curve to estimate blood pressure. Commercially available devices are usually equipped with an electric pump and the bladder contains a pressure sensor. Otherwise, they look similar to the cuff used for manual measurements. Some devices use wrist cuffs, but those are rarely recommended. \cite{BIHS2020} The most significant problem using wrist measurements is the influence of gravity on blood pressure and the accompanying risk of systematic errors when the measurement is not taken at heart level. \cite{Boron2012} This problem is avoided by taking blood pressure at the upper arm.

\paragraph{General Procedure} The procedure is also similar to the manual method. The pressure in the cuff is increased to a level above the expected systolic pressure, which leads to the artery being pushed close. While releasing the pressure slowly through the valve, blood starts flowing through the artery, resulting in small increases in pressure relative to the continued deflation of the cuff. As the cuff deflates further, these relative changes increase to a maximal value before they start decreasing again.\cite{Forouzanfar2015,Drzewiecki1994,Ursino1996} The shape, magnitude or envelope of these oscillations are used in various automatic blood pressure algorithms. Methods that do not use oscillations exist, but are not discussed here. 


\paragraph{Oscillation Extraction} There are two ways the oscillometric waveform (OMW) is extracted from the deflating pressure curve. The first one is filtering. A bandpass filter with a lower cut-off frequency between \SIrange{0.1}{0.5}{\Hz} and an upper cut-off frequency between \SIrange{0.1}{0.5}{\Hz} is recommended. \cite{Forouzanfar2015} Implementations usually use first \cite{Lim2015} to sixth order \cite{Jazbinsek2010} Butterworth filters. They are known for their flat pass-band response and good frequency response. 

The second approach is to use de-trending. It requires to know the beginning of each pulse to be able to fit a line of continuously deflating pressure to the identified points. Its advantage is, that it additionally reproduces an estimated deflation curve. However, it requires additional data, for example, ECG for pulse detection.\cite{Forouzanfar2015}

\begin{figure}[h]
\centering
\includegraphics[width=0.85\textwidth]{figures/oscillations.png}
\caption{A simulated blood pressure extraction. (a) shows the defalting pressure in the cuff and (b) the high-pass filtered oscillations. Source: \cite{Babbs2012} (CC)}
\label{fig:osc}
\end{figure}

Ultimately, from a deflation curve as in the simulated signal in figure \ref{fig:osc} on the top, the oscillations are extracted. The bottom plot in figure \ref{fig:osc} shows oscillations extracted with a thrid order Butterworth bandpass filter with the cut-off frequencies at \SI{0.5}{\Hz} and \SI{5}{\Hz}. \cite{Babbs2012} 


\paragraph{Envelope} The envelope of the OMW is used by basic automatic algorithms. Formation of the envelope is achieved in different ways. The simplest way is to register only local maxima. Similar to that, and most common, is the subtraction of the following through from a local peak or interpolating the curve of local maxima and local minima to subtract them from each other. Some algorithms additionally fit a curve on the obtained oscillometric waveform envelope (OMWE) in an attempt to remove artefacts. \cite{Forouzanfar2014}


\subsection{Maximum Amplitude Algorithm}
The maximum amplitude algorithm (MAA) is based on the assumption that the oscillations are maximal when the pressure in the cuff equals arterial pressure. Accordingly, the recorded pressure at which oscillations are maximal is considered a valid estimate of the MAP \cite{Babbs2012,Geddes1982,Drzewiecki1994}\cite{Ramsey1979}, as long as the compression chamber is kept small \cite{Mauck1980}. To avoid introducing errors, the cuff should always be tightly fit, because air volume in the cuff causes maximum oscillations to be above the true MAP. Hence, Ursio and Cristalli suggest to use the lowest pressure of the plateau of maximal oscillations as the value to assess MAP.\cite{Ursino1996}

A recent publication by Chandrasekhar \textit{et al.} \cite{Chandrasekhar2019} employs a more complex model than the one originally introduced by Mauck \textit{et al.}\cite{Mauck1980}. They conclude that the MAA results in a weighted average of systolic and diastolic BP. According to their model, MAA underestimates MAP for higher blood pressure.

\paragraph{Fixed-Ratio Algorithm} The fixed-ratio algorithm builds on the MAA and is based on the assumption that the systolic and diastolic blood pressure occur at specific fractions of the maximum oscillations before and after its occurrence. Determining SBP and DBP after applying the MAA was tested by Geddes \textit{et al.} in 1982 \cite{Geddes1982}. Geddes recorded Korotkoff sounds while measuring oscillations <to find a ratio of oscillation amplitudes for SBP and DBP. They defined the ratio for the systolic pressure to be 0.5 and the ratio for the diastolic pulse as 0.8. These ratios were found empirically and Geddes acknowledges that the systolic pressure is overestimated and the ratio for the diastolic pulse is not constant for a range of different diastolic pressures.

Later studies tried to find accurate ratios, mostly experimentally with the ratio for SBP usually being determined between 0.45 and 0.73 and the ratio for DBP a bit higher between 0.69 and 0.83.\cite{Drzewiecki1994,Forouzanfar2015} Mathematical models confirmed that a generalised ratio cannot be found. Parameters like the age of the subject and in this regard, the arterial stiffness as well as pulse pressure influence the ratios. \cite{Ursino1996} A higher PP results in a smaller ratio for SBP and larger ratio for DBP. A stiff arterial wall causes volume changes to happen slower. While this has a small effect on the systolic ratio, the diastolic ratio decreases with the stiffness of the artery.\cite{Babbs2012}

% volume changes less rapidly: artery is stiffer!babbs2012
% pulse pressure: higher pp: smaller s ratio, larger d ratio

%- algorithms are usually adjusted to match readings of the auscultatory method, but that has its own shortcomings and might introduce further errors (some use other oscillometric devices which is even worse)

\subsection{Derivative Algorithm}
The derivative algorithm uses the OMWE and plots it against the deflation pressure. The points of the maximal and minimal slope are determined. The pressure point where the derivative of the OMWE is maximal is assumed to be the diastolic BP and where the derivative is minimal is the systolic BP respectively. \cite{Jazbinsek2010,Forouzanfar2015}

Even though mathematical models have confirmed this method to produce valid estimates of both SBP and DBP without the need for empirically obtained ratios, it is extremely vulnerable to noise and therefore not used in practical applications.  \cite{Babbs2012,Chandrasekhar2019}

%TODO move to discussion?
Jazbinsek \textit{et al.} evaluated the fixed-ratio and derivative algorithms. They used a device equipped with ECG and a microphone to validate their results. Various ways to form the envelope were used, including detrending, filtering and even using the low-frequency part of the audio signal of a microphone that recorded Korotkoff sounds. However, to evaluate the fixed-ratio method, they used a commercial device to find average ratios and used the values from the same device as a reference for the evaluation. Their evaluation of the derivative algorithm showed many local extrema as expected. The algorithm was only able to perform by applying additional constraints that bias the measurements significantly. For example, the extrema has to be distanced from the MAP value more than \SI{15}{\mmHg}. \cite{Jazbinsek2010,Jazbinsek2005,Jazbinsek2016}


\subsection{Other Algorithms}
Since the above-mentioned algorithms use only the envelope of the oscillations to determine BP, they discard all the information that lies within the shape of a single pulse. Naturally, scientists have tried to find ways to use this information to improve BP algorithms.

There are a variety of different ideas that have been tested in small scale experiments of mostly with less than 30 test subjects. While they all claim to improve BP measurements compared to the fixed-ratio algorithm, they often lack significant evidence and mathematical validation.


\paragraph{Non-Fixed Ratio}Sapinski \cite{Sapinski1996} proposed a standard algorithm. This algorithm uses ratios to determine the SBP and DBP from MAP as above but calculates the ratios based on the oscillations. The integral of the maximal oscillation is divided by its time period and gives the amplitude of the pulse wave at systolic pressure. The amplitude at diastolic pressure is defined as the difference between the MAP amplitude and the SBP amplitude. This algorithm is based on theoretical assumptions formed form comparisons to the direct method, e.g. that the added oscillation pulse amplitudes at systolic and diastolic pressures equal the amplitude at mean pressure. Sapinski mentions that the average value of systolic oscillation is \SI{40}{\%} of the MAP amplitude and \SI{60}{\%} for the diastolic oscillation, respectively. This part is regularly quoted in other literature, without mentioning that this is not a proposed ratio. Sapinski concludes, that the proposed algorithm does not fulfil the criteria of the AAMI standard compared to the auscultatory method, but does compared to the invasive method. No further literature has been found that validate or contradict these findings.


\paragraph{Pulse Morphology} Specific characteristics of single pulses can be examined to determine blood pressure. Following are four of the most commonly used indices. The definitions are used from Mafi \textit{et al.}\cite{Mafi2011}


\begin{itemize}

\item Stiffness index (SI): The height of the subject ($h$) divided by the time difference between systolic and diastolic peak ($\Delta T$) (equation \ref{eq:SI}) is an indicator for arterial stiffness, but requires to know the person's height.
\begin{equation}
\label{eq:SI}
SI=\frac{h}{\Delta T}
\end{equation}

\item Augmentation index (AI): The difference of systolic ($A_S$) and diastolic peak ($A_D$), divided by the systolic peak, expressed in percentage of pulse pressure (equation \ref{eq:AI}).
\begin{equation}
\label{eq:AI}
AI=\frac{A_S-A_D}{A_S}\times100\%
\end{equation}


\item Reflection index (RI): The of systolic ($A_S$) divided by the diastolic peak ($A_D$), expressed in percentage (equation \ref{eq:RI}).
\begin{equation}
\label{eq:RI}
RI=\frac{A_S}{A_S}\times100\%
\end{equation}



\item $\Delta T/T$ Ratio: The time difference between systolic and diastolic peak ($\Delta T$) is decided by the duration of the pulse ($T$). Both $T$ and $\Delta T$ increase with age.

\end{itemize}

Mafi \textit{et al.}\cite{Mafi2011} plotted the indices in time and used absolute maxima or minima to determine MAP. There is a significant spike where MAP is expected. Using local maxima and minima next to the found MAP to estimate SBP and DBP has less significance. Problematic with this approach is that the reference values were measured by an Omron device with an unknown implementation of oscillometric blood pressure. Because the device did not display the measured MAP, the MAP reference was calculated using equation \ref{eq:MAP}, which is known be wrong using accurate values of SBP and DBP and likely delivers arbitrary values when using values from an automatic device.

%TODO discussion?
Another implementation from Mafi \textit{et al.}\cite{Mafi2012} is taking the '1st' derivative of each pulse and taking the maximal value of that to form another curve that looks similar to the one used in MAA. This curve is then treated equally to the fixed-ratio algorithm to determine MAP, SBP and DBP. The ratios are determined experimentally using reference measurements. Additionally, the MAP is again calculated using the faulty equation \ref{eq:MAP} and the 18 test subjects aged between 24 and 68 are healthy and have no history of cardiovascular disease. The authors argue that the shape is less sensitive to noise than the amplitude and therefore, their approach is more robust than MAA.


\paragraph{Model-Based Algorithms} These algorithms use a predicted OMWE and fit it to the observed one to determine the characteristics. While they do consider factors that the standard algorithms discard, such as arterial stiffness and pulse pressure, they are vulnerable to artefacts that are not considered by the model. They could be used to validate algorithms. \cite{Babbs2012}

\paragraph{Neural Networks} The features extracted from above, and more, are often used in neural networks (NN). However, it has been shown, that the indices can reliably only be used to determine the MAP. Moreover, some of the features are highly dependant on external factors. For example, the exact deflation rate influences the duration of the OMWE (proposed by Lee \textit{et al.}\cite{Lee2013}) and the constraint of \SIrange{2}{3}{\mmHg/\second} given by Lim \textit{et al.}\cite{Lim2015}, who implemented a NN using the proposed features, seems hardly enough. Similarly, the used filter by Lim \textit{et al.}, a first-order bandpass Butterworth filter with cut-off frequencies of \SI{0.5}{\Hz} and \SI{5}{\Hz}, is likely to remove valuable information from the pulses.


