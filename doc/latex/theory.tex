\chapter[theory]{Theory} %TODO maybe call this chapter background
This chapter discusses the findings of researching existing and proposed algorithms for determining systolic and diastolic blood pressure through the oscillometric method.

A brief section at the end of this chapter explores algorithms beyond the scope of this project that have been proposed in recent literature.

\section{What is Blood Pressure?}
Blood pressure is an important bio-medical measurement and often used for  diagnostics in cardiovascular diseases. Most commonly, systolic and diastolic blood pressure are mentioned in pairs, but what do those numbers mean?

The heart is a pump that has two phases. When the ventricle is relaxed, the heart fills up and the diastolic blood pressure (DBP) is observed. When the heart contracts, it pushes the blood through the arterial system and the systolic blood pressure (SBP) is observed. A third characteristic is the difference between SBP and DBP, the pulse pressure (PP). Finally, the mean arterial pressure (MAP) is defined as the average pressure in the artery. Figure \ref{fig:BP} shows how these pressure values are connected in one heart beat. MAP is the area underneath the blue curve divided by the time of one pulse, indicated by the yellow cross-hatched area. \cite{Boron2012}\paragraph{}

\begin{figure}[]
\centering
%\includegraphics[scale=0.125]{GlaLogo.pdf}
\caption{Blood pressure characteristic values. Source: \cite{Boron2012}}
\label{fig:BP}
\end{figure}


Blood pressure can be measured with invasive and non-invasive methods. The most accurate methods are invasive, but require professional expertise because a catheter has to be injected into the blood vessel. Subsequentially, non-invasive methods are more common because they are more convenient to use. With automated methods, taking blood pressure at home has become convenient and easy. However, the inaccuracies of these measurements are often ignored or even unknown.

\section{Manual Blood Pressure Measurement}
The manual method of measuring blood pressure relies on a sphygmomanometer and listening to the heart or Korokoff sounds (named after their discoverer) with a stethoscope. This is called the \emph{auscultatory method}. \emph{Palpation} is another method where the pulse is felt at the wrist. It only allows detection of the SBP. The ausculatory method is described below because it provides recommendations that are true for all blood pressure measurements using a cuff.

The sphygmomanometer is shown in figure \ref{fig:sphy}. It consists of a inflatable cuff that is connected to a rubber pump and a scale that indicates the pressure in the cuff in \SI{}{\mmHg}. The pump is equipped with a valve that can be opened to release air.


\SI{}{\mmHg} is a unit to descibe pressure. \SI{1}{\mmHg} is defined as the pressure generated by a column of mercury of \SI{1}{\mm} heigth. Millimeters of Mercury is not part of the International System of Units (SI), but was defined by it before 2019 as \SI{1}{\mmHg} = \SI{133.322}{\Pa} using standard gravitiy.\cite{SI2008} Since 2019, \SI{}{\mmHg} is no longer included in the SI brochure. However, it is still widely used in the medical field.


\begin{figure}[h]
\centering
%\includegraphics[scale=0.125]{GlaLogo.pdf}
\caption{Picture of a sphygmomanometer showing the cuff, inflation pump and mercury meter.}
\label{fig:sphy}
\end{figure}

Auscultatory blood pressure measurements are often considered the standard and what automatic methods are compared against. \cite{Sapinski1996}


\subsection{Auscultatory Method}
It is important that the cuff size is appropriate for the patient's arm. The rubber bladder inside the cuff should cover more than \SI{80}{\%} but less than \SI{100}{\%} of the arm's circumference.

The cuff is completely deflated and tightly fit around the upper arm of the patient. The center of the bladder should be over the brachial artery. The stethoscope is to be placed over the artery between the cuff and the patient's elbow and should not touch the cuff. This could influence the reading because it puts additional pressure on the artery and can interfere with hearing the sounds.\cite{NHS2019}\cite{Reeves1995}

The valve at the pump is completely closed and the cuff is inflated to a pressure of about \SI{30}{\mmHg} above the expected systolic pressure. This causes the artery to flatten and not let any blood through. Subsequently, the valve is slightly opened to slowly deflate the cuff at a rate of approximately \SI{3}{\mmHg/\second}. The artery will open and let small amounts of blood through. Systolic pressure is read when a pulse is first heard while deflation is continued. Diastolic pressure is read when the sounds disappear completely. Ultimately, blood can flow freely thorugh the artery. After making sure that no further sounds can be heard by deflating for at least another \SI{10}{\mmHg}, the valve is opened completely to rapildly empty the cuff.   \cite{NHS2019,Reeves1995}

There are some discrepancies in literature on how to define the point of the diastolic blood pressure. Some references define it as when the Korokoff sounds dissappear completely. \cite{NHS2019,Reeves1995} Others, predominantly older ressouces, define it as the point where the sounds are muffled.\cite{Boron2012} Here, the former is assumed, because most current literature does and some literature suggests that the auscultatory method overestimates diastolic blood pressure. \cite{Chandrasekhar2019}


\section{Automatic Blood Pressure Measurement}

Most modern automatic devices use the oscillometric method detmine blood pressure. They are used both at home and in professional environments.

Automatic blood pressure monitors measure pressure in a cuff similar to the one used in the manual method and use the small pressure changes (oscillations) extracted from the deflation curve to estimate blood pressure. Commercially available devices are usually equipped with an electic pump and the bladder contains a pressure sensor. Otherwise, they look similar to the cuff used for manual measurements. Some devices use wrist cuffs, but those are rarely recommended. \cite{BIHSOC2020} The most immediate problem with wrist measurements is the influence of gravity on blood pressue and the accompaning risk of systematic errors when the measurement is not taken at heart level. \cite{Boron2012} This problem is avoided by taking blood pressure at the upper arm.

\paragraph{General Procedure} The procedure is also similar to the manual method. The pressure in the cuff is increased to a level above the expected systolic pressure, which leads to the artery being pushed close. While releasing the pressure slowly through the valve, blood starts flowing through the artery, resulting in small increases in pressure realative to the continuous deflation of the cuff. As the cuff deflates further, these relative changes increase to a maximal value before they start decreasing agian.\cite{Forouzanfar2015,Drzewiecki1994,Ursino1996} The shape, magnitude or envelope of these oscillations are used in various automatic blood pressure algorithms. Methods that do not use oscillations exist, but are not discussed here.


\subsection{MAA}


MAP at max osc:\cite{Geddes1982,Drzewiecki1994,Babbs2012}\cite{Ramsey1979} provided compression chamber is kept small \cite{Mauck1980}
doubt, but still close: weighted av\cite{Chandrasekhar2019}
lowest pressure of the platau of maximal oscillations \cite{Ursino1996}


a lot of $new$ algorithms can be brought back to this method, they use NN or models to improve the ratios, but in the end, it is the same.
\subsection{Derrivative Algorighm}
then go into the different algorithms that were proposed.


MAA (fixed ratio)
Derrivative
pulse morphology
model based
NN
PTT

\cite{Chandrasekhar2019}
fixed rato: overestimates systolic and underestimates diastolic

mention artifact remvoal:
what artifacts: high frequency noise, movement, muscle contractions
 --> how to make sure only artifacts and not information is remvoed?


Determing blood pressure by using bandpass filterd pressure data (oscillations) was tested by Geddes et al. in 1982 \cite{Geddes1982}. Geddes recorded Korokoff sounds while measuring oscillations in order to find a ratio.

The mean arterial pressure (MAP) is generally described as the pressure at which oscillation amplitudes are maximal. Geddes is determining ratios of the MAP to find the systolic and diastolic blood pressure. He defines the ratio for the systolic pressure to be 0.5 and the ratio for the diastolic pulse as 0.8. These ratios were found in an experimental way and Geddes acknowledges that the systolic pressure is overestimated and the ratio for the diastolic pulse is not constant for a range of different diastolic pressures.

Later studies tried to find accurate ratios, mostly experimentally. Mathematical models confirmed that a generalised ratio cannot be found. Parameters like the age of the subject and in this regard, the arterial stiffness influence the ratios. \cite{Babbs2012}
