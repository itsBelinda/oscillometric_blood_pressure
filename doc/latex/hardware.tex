\chapter{Hardware}\label{cp:hw}
The hardware used in this project was provided. This chapter describes the basic setup upon which the developed application is built. 

\paragraph{Overview}\mbox{}\newline
Figure \ref{fig:hw} shows a schematic view of the hardware setup. The computer running the application is connected to the data acquisition device, the USB-DUX Sigma, through an USB-port. On channel 0 of the device, the pressure sensor is connected through a voltage divider with a backup capacitor. The voltage divider distributes the voltage roughly 2:1. The software should be calibrated to account for the actual values.

\paragraph{Pressure Sensor}\mbox{}\newline
The pressure sensor measures absolute pressure. It operates on \SI{0}{\volt} supplied by the USB-DUX Sigma device. It outputs \SI{1}{\volt} per \SI{50}{\kilo\pascal} measured pressure. The output range is from \SIrange{0}{4}{\volt}. The characteristic values are summarised in table \ref{tbl:sensor}.


\paragraph{Analogue to Digital Converter}\mbox{}\newline
The provided analogue to digital converter (ADC) is part of a USB-DUX Sigma device. The device offers multi-channel in and outputs. However, only a single ADC channel was used. 
The used channel provides 24-bit resolution at a sampling rate of \SI{1}{\kilo\hertz}. The input range of the device is \SIrange{-1.375}{1.375}{\volt}, which is not large enough for the output of the pressure sensor to be directly connected. Therefore, the above mentioned voltage divider is required.

Table \ref{tbl:sigma} summarises the characteristic values of the USB-DUX Sigma device. 
