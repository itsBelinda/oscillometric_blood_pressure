\chapter{Algorithm}\label{cp:algo}

This chapter explains the general principles behind the algorithm and separates the logic from the implementation, which will be discussed in a separate chapter, \myref{cp:sw}. 

The realised algorithm is a fixed-ratio algorithm. It analyses the oscillations to find MAP form the maximum amplitude as described in \myref{sec:MAA}. It estimates the position of the systolic and diastolic BP as where the amplitudes are at a specific ratio of the maximum. The systolic BP is evaluated while the oscillation amplitudes increase and the diastolic BP while they decrease, respectively.


\begin{figure}[ht]
\centering
\includegraphics[width=\textwidth]{figures/algo_overview.pdf}
\caption{A sample dataset is shown in the top plot and divided into three parts. The blue portion is where oscillations are analysed. These are shown in the bottom plot.}
\label{fig:algoOverview}
\end{figure}

The top plot in figure \ref{fig:algoOverview} shows one data set and how it is divided into different steps for processing. The first part (green) is when the user pumps up the cuff pressure. The second part (blue) is the deflation. Here, the oscillations, shown in the bottom plot, are recorded and analysed. Once these oscillations have increased to a maximum and decreased again to a certain value, the last step (red) is simply the deflation of the cuff. The core part of the algorithm works on the blue part of the data.



\section{Preprocessing}
The data is always filtered. Additionally, preprocessing decides which stage of the dataset is active. Concerning figure \ref{fig:algoOverview}, stage one is green, stage two is blue, and stage three is red. Figure \ref{fig:algoP} shows the basic diagram of data processing.  The core algorithm processes only the deflating and filtered data.  

\subsection{Filtering}\label{sec:filt}
Unwanted noise is removed from the data by a filter. Oscillations are extracted by a second. Figure \ref{fig:algoP} below shows the simple cascade of a low-pass (LP) and high-pass (HP) filter, resulting in a bandpass(BP) filtered output signal, the oscillogram. The results of both filter outputs are used in the algorithm.


\begin{figure}[ht]
\centering
\includegraphics[width=\textwidth]{figures/filter1.pdf}
\caption{Filter.}
\label{fig:filters}
\end{figure}
%All data is filtered. 

From the research done in chapter \myref{cp:theory}, it is known that most algorithms use band pass filters of low orders up to \nth{6} and with cut-off frequencies between \SIrange{0.1}{0.5}{\Hz} and \SIrange{5}{20}{\Hz} \citep{Forouzanfar2015}. The implemented algorithm does not rely on pulse shape and can, therefore, use relaxed constraints. Best results were achieved with a lower cut-off frequency of \SI{0.5}{\Hz}. The upper cut-off frequency was fixed \SI{10}{\Hz}. Generally, there is very little activity between \SIrange{5}{20}{\Hz}, but because no experiments could be done with test subjects that have higher blood pressure, a mean value was chosen. 

The implemented filters are shown in figure \ref{fig:filters}. The low-pass filter has a small delay in the impulse response, and the phase response shows that for very low frequencies, there is a slightly higher delay. The frequency response shows a steep slope after \SI{10}{\Hz} and suppression of more than \SI{20}{\decibel} of the spectrum above \SI{20}{\Hz}. The HP filter expectedly lets the impulse pass. In the frequency response, there is a small overshoot but otherwise a good suppression of DC. The bottom left plot in figure \ref{fig:filters} shows that the HP filter has a significant delay of about \SI{125}{\milli\second} for frequencies around \SI{1}{\Hz}, which could potentially cause inaccuracies. Section \myref{sec:impfilt} will explain why this is not a problem.


\subsection{Stage Detection}
The LP filtered data is analysed by the preprocessing step to detect if the second stage, the deflation, is active. Figure \ref{fig:algoP} shows how the preprocessing step then passes both the LP and BP filtered data to the core algorithm, symbolised by a switch. 

\begin{figure}[ht]
\centering
\includegraphics[width=0.7\textwidth]{figures/algo_process.pdf}
\caption{The figure shows the simple basic data processing flow. The data is acquired and passed first to a low-pass filter (\SI{10}{\Hz}) and then to a high-pass filter (\SI{0.5}{\Hz}), resulting in a bandpass output. The first stage of the data processing only checks if the pressure in the cuff is high enough to start deflation (stage 2).}
\label{fig:algoP}
\end{figure}

In the first stage, the data is checked if the pressure in the cuff is high enough to start deflation. If so, the preprocessing switches to the second stage. This value is configurable. The default value is \SI{180}{\mmHg}. Figure \ref{fig:algoP} shows how the decision is made based on the LP filtered data.  

Once the second stage is active, the preprocessing checks the core algorithm if it is done, in figure \ref{fig:algoP}, this is indicated by the arrow from the algorithm back to preprocessing. Additionally, the data is checked if it is below a value of \SI{20}{\mmHg}, where taking blood pressure makes no sense, and the measurement is potentially cancelled.

\newpage


\section{Deflation}\label{sec:Deflation}
Figure \ref{fig:algoExplain} shows data recorded during the deflation stage. \circled{1} shows the low pass filtered data, which is used in the core algorithm as a reference. \circled{2} is the BP filtered data upon which the algorithm is performed. 

\begin{figure}[ht]
\centering
\includegraphics[width=0.9\textwidth]{figures/algorithm_example_annotated.pdf}
\caption{Algo explanation.}
\label{fig:algoExplain}
\end{figure}



\subsection{Extrema Detection}
\paragraph{Maxima} The next step, indicated with \circled{3}, is to detect the maxima. There are local maxima in the troughs between two oscillations. To avoid detecting these, a value needs to have a minimal pronunciation of \SI{0.25}{\Delta\mmHg} to be considered as a maximum. This limits how quickly the first oscillations can be recognised but does not influence the algorithm because they are less than \SI{10}{\percent} of the expected maximal amplitude.


\begin{figure}[ht]
\centering
\includegraphics[width=0.95\textwidth]{figures/maxima_issue.pdf}
\caption{Algo explanation.}
\label{fig:maxDetectIssue}
\end{figure}


A sample is identified as a local maximum, due to the previous and following value being lower. Next, is checked if the current and last detected maxima result in a valid heart rate given by a configurable range. The default range is \SIrange{50}{100}{beats/\minute}. Also, if the last maximum was less than \SI{300}{\milli\second} ago, only the larger of the two is kept, and the lower is discarded.

This also solves two concerns that were observed in the data and are shown in figure \ref{fig:maxDetectIssue}. The first, shown on the left-hand side, is if the bump in the rising edge of the oscillations results in a detected maximum. It often happens if the deflation is not steady or the arm is moved. When oscillations are small and rising, these can lead to wrong detections. Either these extrema are irrelevant because they occur before systolic BP, or they will be discarded because they lead to errors.

The second issue is shown in figure \ref{fig:maxDetectIssue} on the right. It occurs mainly decreasing oscillations and if the filtering is not optimal. The algorithm successfully handles these double maxima, but it might also influence the detected DBP value. 

\paragraph{Minima} Minima detection, indicated by \circled{4} in figure \ref{fig:algoExplain}, is considerably more difficult, especially at the beginning of oscillations as shown in figure \ref{fig:oscMini}. Because there has to be a minimum in between two maxima, minima detection is simplified. After detecting two maxima, the lowest value between them is recorded as the minima.

\begin{figure}[ht]
\centering
\includegraphics[width=0.95\textwidth]{figures/minima_issue.pdf}
\caption{Algo explanation.}
\label{fig:oscMini}
\end{figure}

\subsection{Heart Rate Detection}\label{sec:HR}
Heart rate is already part of the extrema detection described above. Therefore, the current heart rate is detected after each valid maxima starting from the second one. Because each pulse stems from a heartbeat, each peak represents one. By calculating the time between the last two maxima, a heartbeat is estimated using equation \ref{eq:HR} below.

\begin{equation}
\label{eq:HR}
pulse=\frac{60s}{t_{curMax} - t_{lastMax}}
\end{equation}

After finishing the measurement, the average heart rate is calculated from all validly detected maxima.


\section{Oscillometric Waveform Envelope}
When the detected maxima, marked by the red triangles in \ref{fig:algoExplain}, have reached their highest amplitude and declined again, the oscillometric waveform envelope (OMWE) can be calculated. 

This condition is estimated by taking the largest recorded maximum and multiplying it with the ratio for the diastolic blood pressure minus a hysteresis of $0.3$ (equation \ref{eq:cutoff}. If the last recorded maxima is below this value, the extrema detection part of the algorithm (\circled{3} and \circled{4} in figure \ref{fig:algoExplain}) can be concluded. To avoid wrong detections, the largest maxima needs to be at least \SI{1.5}{\delta\mmHg}. Otherwise the condition can be detected when oscillations first start occurring and are small with large variations. 

\begin{equation}
\label{eq:cutoff}
A_{cutoff}=A_{Max}\times (r_{DBP} - 0.3)
\end{equation}

\subsection{OMWE Calculation}
The following approaches were tested to calculate the OMWE from the determined extrema:

\begin{enumerate}[noitemsep]
\item Only considering the maxima.
\item Subtracting the following minimum from each maximum.
\item Subtracting the preceding minimum from each maximum.
\item Interpolating between maxima and minima to find the corresponding values and subtracting them. 
\end{enumerate}

1. works decently but discards the information in the minima. 2. works well for pressure in the systolic range because minima follow quickly after maxima. Vice versa, 3. works well in the diastolic range. 4. requires the most computations, but is the least sensitive to noise, because it takes the time difference between minima and maxima into consideration. 

When referring back to figure \ref{fig:algoExplain}, \circled{5} shows how the OMWE was calculated using the \nth{4}approach and calculating a value for each of the maxima and minima separately. 



\begin{figure}[ht]
\centering
\includegraphics[width=0.75\textwidth]{figures/omwe_exp.pdf}
\caption{Algo explanation.}
\label{fig:interpol}
\end{figure}


Figure \ref{fig:interpol} explains how interpolation is used to calculate an interpolated maximum at the time where a minimum occurs (\circled{1}) to then be able to subtract the minimum from the maximum. Inversely, A minimum is interpolated between two measured minima where a maximum occurs (\circled{2}). The resulting plot is displayed in figure \ref{fig:algoExplain} at the bottom. Note that the values are normalised by the maximal value.


\subsection{Determination of Blood Pressure Values}
After calculating the OMWE, the last step is to determine the BP values from it. As explained in chapter \myref{cp:theory}, MAP occurs where the OMWE has its maximum. The time, when the maximum is recorded, is compared to the pressure at that time. The green arrows at \circled{6} in figure \ref{fig:algoExplain} indicate this. To account for the oscillations in the deflation curve, the pressure is averaged throughout the average heartbeat centred around the detected time. The average heartbeat is determined from the measurement, as explained in section \myref{sec:HR}. In the example above, this results in a MAP of \SI{71}{\mmHg}.


Likewise, to determine SBP and DBP, the specific ratios of the maximal oscillations are searched in the OMWE. The example in figure \ref{fig:algoExplain} uses a ratio of 0.5 for systolic BP and 0.8 for diastolic BP.

The systolic BP is determined by stepping backwards in time from the maximal oscillations and the diastolic BP by stepping forwards in time, respectively. Once the value falls below the specific ratio, interpolation is used to find the time where the exact ratio would have occurred. Using the same approach as for the MAP, the specific values found with \circled{6} for the data in figure \ref{fig:algoExplain} is \SI{95}{\mmHg} for SBP and \SI{67}{\mmHg} for DBP.


\subsubsection{Implications of Filter Delays}\label{sec:impfilt}
As pointed out in section \myref{sec:filt}, the original data is delayed by first the LP and then the HP filter. Since the oscillogram is based on the LP filtered data, the delay of the first filter (LP) is irrelevant. The second filter (HP) has up to \SI{125}{\milli\second} delay for certain frequencies. Figure \ref{fig:algoDetail} shows the oscillations superimposed over the deflating data. At a deflation rate of \SI{3}{\mmHg/\second}, the error could be \SI{0.375}{\mmHg}. The calculated pressure is averaged throughout an average heart rate to account for the oscillations in the deflating data. Compared to the inaccuracies of the algorithm in general, this is neglectable. 


\begin{figure}[ht]
\centering
\includegraphics[width=0.85\textwidth]{figures/algo_detail.pdf}
\caption{Algo explanation.}
\label{fig:algoDetail}
\end{figure}
